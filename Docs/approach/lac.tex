\documentclass[paper=a4, fontsize=12pt]{scrartcl}
\usepackage[utf8]{inputenc}
\usepackage{amssymb}
\usepackage{amsmath}
\usepackage{mathtools}
\begin{document}
\begin{center}
\Large{\textbf{OLAC}\\
- \\
\textbf{Label Buying Decision Engine}}\\[5mm]
\normalsize
\end{center}
\textbf{Problem setting:}\\
Compute the optimal allocation of labels acquisitions that maximize, under resource constraints, long run utility.\\

\subsubsection*{Definitions:}
\begin{itemize}
\item Let $\mathbf{X} \in \mathbb{R}^{nm}$ be the observed data.
\item Let $\textbf{f} \in \mathbb{R}^{m}$ be the vector of features.
\item Let $\textbf{H} \in \{0,1\}^{nk_{t}}$ be the cluster mask.
\item Let $\textbf{y} \in \{0,1\}^n$ be the vector of true labels.
\item Let $\theta \in \{0,1\}^n$ be the vector of acquired labels.
\item Let $\mathcal{K}_{t}$ be the set of clusters of size $k_{t}$ at time $t$.
\item Let $\textbf{k} \in \{1,2,\ldots,n\}^{t}$ be the vector of optimal number of clusters.
\item Let $\textbf{u} \in \mathbb{R}^{n}\sim LN(\mu, \sigma)$ the vector of utility earned by detecting fraud drawn from a log-normal distribution.
\end{itemize}
\subsubsection*{Information K-means}
For each block of time $t$, $t \in \mathbb{Z}$, the newly observed data is compared to the historical data and evaluate whether the number of clusters needs to be adjusted.\\
Let $k_{t} = g(\textbf{f})$; where $g$ is a function that approximates the optimal number of clusters.
Let $\mathcal{K}_{t} = f(k_{t}, \textbf{X})$ where $f$ is the K-means algorithm that approximates the optimal centroid location and assignment of data points.\\

\noindent\textit{cluster assignment}:
\begin{equation}
K_{\eta}^{(q)} = \{x_i~:~\lVert x_i - m_{\eta} \rVert^{2} \leq \lVert x_i - m_{j} \rVert^{2}~\forall~j, 1 \leq j \leq k\}
\end{equation}
\textit{centroid update}:
\begin{equation}
m_{\eta}^{(q + 1)} = \frac{1}{\lvert K_{\eta}^{(q)}\rvert}\sum_{x_{i} \in K_{\eta}^{(q)}} x_{i}
\end{equation}

\subsubsection*{Utility function}
Utility consists of the cost of buying a label, $c \in \mathbb{R_{+}}$, the revenue gained by detecting fraud, $u \in \mathbb{R_+}$ with the assumption that the
utility gained from determining fraud is greater than the cost, $c < u$. Additional elements can be added such as cost of true negatives, missing fraud, or additional costs for investigating false positives.
\begin{equation}
U(\textbf{u}, \mathbf{y}, \theta, c) = \textbf{u} \times (\textbf{y} \times \theta) - c \theta = \sum_{i=0}^{n}u_{i}y_{i}\theta_{i} - c\theta_{i}
\end{equation}

\subsubsection*{Revenue}
In order to determine the monetary gain of buying a label in a particular cluster we calculate the expected revenue of allocating all labels to a particular cluster. The matrix $\textbf{H} \in \{0,1\}^{nk}$ contains the masks for each cluster, $K_\eta \subset \mathcal{K} \text{ where } $, i.e.:
\begin{equation}
h_{i,\eta}=
\begin{dcases}
1,& \text{if } \textbf{x}_{i} \in K_\eta\\
0, & \text{otherwise}
\end{dcases}
\end{equation}
where $i \in \{0,\ldots,n-1\},~\eta \in \{1, \ldots, k\}$.\\

The expected revenue for cluster $\eta$ is defined as:
\begin{equation}
\pi_{\eta} = \textbf{h}_{\eta} \cdot \mathop{\mathbb{E}}[U(\textbf{u}, \mathbf{y}, \theta, c)] = \textbf{h}_{\eta} \cdot (\mathop{\mathbb{E}}[\textbf{u}] \times (\textbf{y} \times \theta) - c\theta)
\end{equation}
\end{document}

